\documentclass{article}
\usepackage{inconsolata}
\usepackage{xspace}
\usepackage{hyperref}
\usepackage{cleveref} % use this to avoid the "typos" in Sections/Figure references
\usepackage{siunitx}
\sisetup{
    detect-all,               % Automatically detect surrounding font styles (bold, italic, etc.)
    output-decimal-marker=.,  % Use a period as the decimal marker (default is locale-specific)
    separate-uncertainty=true,% Separate uncertainty with ± symbol
    multi-part-units=single,  % Use single units for multi-part values
    per-mode=symbol,          % Use '/' for units like m/s
    exponent-product=\cdot,   % Use a dot (·) for scientific notation
    output-complex-root=\text{j}, % Use 'j' for imaginary unit
    range-phrase=--,          % Use an en-dash for ranges (e.g., 10--20)
    range-units=single,       % Display the unit only once in ranges
    group-digits=integer,     % Group digits in integers
    group-separator={\,},     % Use a thin space for digit grouping
    group-minimum-digits=4,   % Start grouping for numbers with 4+ digits
    retain-explicit-plus=true % Keep '+' sign for positive numbers
}

\newcommand{\cd}[1]{\texttt{\small #1}\xspace} % code
\newcommand{\verbcomment}[1]{\textbf{\textcolor{gray}{#1}}} % colored comments for verbatim environments
\newcommand{\fncd}[1]{\texttt{\footnotesize #1}\xspace} % code in footnotes
\newcommand{\pptext}[1]{\medskip\noindent \textit{#1.}\xspace} % paragraph text
\newcommand{\diam}{\hfill$\diamond$} % diamond symbol
\newcommand{\ie}{i.e.,\,}
\newcommand{\eg}{e.g.,\,}

\crefname{section}{Sect.}{Sect.}
\Crefname{section}{Sect.}{Sect.}
\crefname{figure}{Fig.}{Fig.}
\Crefname{figure}{Fig.}{Fig.}
\crefname{table}{Table}{Table}
\Crefname{table}{Table}{Table}

% Framed-box for definitions
\newcommand{\colbox}[1]{
\medskip
\setlength{\fboxsep}{0.5em}
    % uncomment for a colored box
    % \fcolorbox{babyblueeyes}{aliceblue}{
    \fbox{
    \centering
    \begin{minipage}{0.9\textwidth}
        \textit{#1}
    \end{minipage}}
\medskip
}

\begin{document}

\section{Section Title}
\label{sec:section-title}

\subsection{Subsection title}
\label{sec:subsection-title}

\pptext{Crossreference}

\verb!\cref{sec:section-title}! to refer to \cref{sec:section-title}.

\Cref{sec:subsection-title}, \verb!\Cref{sec:subsection-title}! when the reference is at the beginning of a sentence.

\verb!\cref*{sec:section-title}! to avoid creating hyperlinks $\longrightarrow$ \cref*{sec:section-title}.
\medskip

\pptext{Code}

\verb!\cd{code}! to use the inconsolata $\longrightarrow$ \cd{code}.

\verb!\fncd{code}! to use the inconsolata in footnotes $\longrightarrow$ \fncd{code}.
\medskip

\pptext{Diamond}

Use \verb!\diam! to add a diamond symbol at the end of a line: (used for examples sections) \diam
\medskip

\pptext{Colored/Framed box}

\colbox{Lorem ipsum dolor sit amet, consectetur adipiscing elit. Fusce tempor tincidunt commodo. Nulla suscipit molestie faucibus. Phasellus metus ante, consectetur sit amet tincidunt a, vehicula sit amet nulla. In euismod tempor sem in dictum. Praesent odio lorem, laoreet tincidunt varius nec, blandit vitae tortor. In accumsan a libero sed aliquam. Aenean quis commodo nisi. Pellentesque ut eros a orci tristique pharetra in non nunc.}

\pptext{SIUnit}

\verb!\SI{10}{\meter}! $\longrightarrow$ \SI{10}{\meter}.

\verb!\num{10}! $\longrightarrow$ \num{10}.

\verb!\unit{\kilo\gram\metre\per\square\second}! $\longrightarrow$ \unit{\kilo\gram\metre\per\square\second}.

\end{document}
